\chapter{Implementacija i korisničko sučelje}
		
		
		\section{Korištene tehnologije i alati}
		Pisana komunikacija u timu odvijala se putem aplikacije WhatsApp\footnote{https://www.whatsapp.com/}, a sastanci putem Google Meeta\footnote{https://meet.google.com/} i Discorda\footnote{https://discord.com/}. WhatsApp je besplatna mobilna aplikacija koja služi za razmjenu poruka, fotografija i videozapisa putem mobilnog interneta pametnim telefonom. Google Meet je usluga videokomunikacije koju je razvio Google, dok je Discord VoIP aplikacija koja omogućava komunikaciju glasom, videom i tekstom.
		
		UML dijagrami napravljeni su alatom Visual Paradigm Online\footnote{https://online.visual-paradigm.com/}. Visual Paradigm Online je online alat za crtanje dijagrama i njihovu pohranu u web-pregledniku koji omogućava istovremeni rad više korisnika u stvarnom vremenu. Dijagrame možemo preuzeti u raznim formatima (.png, .jpeg, .pdf i dr.).
		
		Za vizualizaciju stranice korištena je Figma\footnote{https://www.figma.com/}, besplatan alat za UI i UX dizajn, uređivanje i izradu	prototipova te generiranje koda dostupan na webu ili u obliku desktop aplikacije.
		
		Izvornim kodom upravljano je sustavom Git\footnote{https://git-scm.com/}. Git je distribuirani sustav za upravljanje različitim	verzijama podataka (programskog koda, teksta i dr.). Sastoji se od udaljenog repozitorija koji se nalazi na nekoj Git platformi u oblaku i od lokalnih kopija tog repozitorija na računalima korisnika koji rade na projektu. Udaljeni repozitorij ovog projekta dostupan je na web platformi Gitlab\footnote{https://about.gitlab.com/}.
		
		Kao razvojno okruženje korišten je Visual Studio Code\footnote{https://code.visualstudio.com/}. Visual Studio Code
		je uređivač izvornog koda koji je razvio Microsoft za Linux, Windows i Mac OS platforme. Uključuje podršku za uklanjanje pogrešaka, isticanje sintakse, inteligentno dovršavanje koda, isječke, refaktoriranje koda i ugrađeni Git.
		
		Osim VSCode-a, koristili smo i JetBrains WebStorm\footnote{https://www.jetbrains.com/webstorm/} i JetBrains DataGrip\footnote{https://www.jetbrains.com/datagrip/}. WebStorm je integrirana razvojna okolina za JavaScript i povezane tehnologije, a DataGrip detektira moguće \textit{buggove} u kodu i predlaže najbolje opcije za njihovo ispravljanje.
		
		Cijeli sustav pisan je jezikom TypeScript\footnote{https://www.typescriptlang.org/} koji je proširenje jezika JavaScript(JavaScript s tipovima), skriptnog programskog jezika koji omogućava interakciju korisnika s web-stranicom.
		
		Za frontend smo koristili React\footnote{https://reactjs.org/} i Chakra UI\footnote{https://chakra-ui.com/}. React je knjižnica koja služi za izgradnju korisničkog sučelja ili UI komponenti, a Chakra je jednostavna modularna bibloteka komponenata	u kojoj se nalaze blokovi potrebni za izgradnju React aplikacija.
		
		 PostgreSQL\footnote{https://www.postgresql.org/} baza podataka spremljena je na $Heroku^{14}$	koji je ujedno i server. PostgreSQL je besplatni i relacijski sustav upravljanja bazom podataka otvorenog koda dizajniran za upravljanje nizom radnih opterećenja, od pojedinačnih strojeva do skladišta podataka ili web-usluga s mnogim istovremenim korisnicima. Heroku je platforma u oblaku, točnije platforma kao usluga (engl. Platform as a Service, skraćeno PaaS), što znači da korisnici	na nju postavljaju ili na njoj izraduju aplikaciju koja će se na njoj i izvršavati.
		
			\eject 
		
	
		\section{Ispitivanje programskog rješenja}
			
			 \textit{Ispitivanje implementiranih funkcionalnosti provedeno je na dvije razine - kroz nezavisno ispitivanje individualnih komponenti, a zatim i kroz ispitivanje čitavog sustava s prikazom odabranih ispitnih slučajeva. Korišteni su strukturirani načini ispitivanja - unit testovi za ispitivanje komponenti te Selenium testovi za ispitivanje sustava. Ispitani su rubni uvjeti i temeljna funkcionalnost.}
			
			\subsection{Ispitivanje komponenti}
			\textit{Provedeno je ispitivanje jedinica (engl. unit testing) nad razredima koji implementiraju temeljne funkcije. Tražene su pogreške u algoritmima, podacima i sintaksi. Testovi su prikazani u nastavku.}
			
			\noindent \underbar{\textbf{Testni Slučaj 1}}
			\begin{lstlisting}
				const mocks = [
					{
					  request: { query: me },
					  result: {
						data: {
						  me: null,
						},
					  },
					},
				  ];
				  
				  describe("Actual", () => {
					it("renders a heading", () => {
					  render(
						<MockedProvider mocks={mocks} addTypename={false}>
						  <ActualEarthquakesPage />
						</MockedProvider>
					  );
				  
					  const heading = screen.getByText("Loading...");
					  expect(heading).toBeInTheDocument();
					});
				  });
			\end{lstlisting}

			\noindent \underbar{\textbf{Testni Slučaj 2}}
			\begin{lstlisting}
				const mocks = [
					{
					  request: { query: me },
					  result: {
						data: {
						  me: null,
						},
					  },
					},
				  ];
				  
				  const earthquakesMock = {
					request: { query: earthquakes, variables: { archived: true } },
					result: {
					  data: {
						earthquakes: [],
					  },
					},
				  };
				  
				  describe("Archived", () => {
					it("renders a loading", () => {
					  render(
						<MockedProvider mocks={mocks} addTypename={false}>
						  <ArchivedEarthquakesPage />
						</MockedProvider>
					  );
				  
					  const heading = screen.getByText("Loading...");
					  expect(heading).toBeInTheDocument();
					});
				  });
			\end{lstlisting}

			\bigskip

			\noindent \underbar{\textbf{Testni Slučaj 3}}
			\begin{lstlisting}
				const mocks = [
					{
					  request: { query: me },
					  result: {
						data: {
						  me: null,
						},
					  },
					},
				  ];
				  
				  describe("Home", () => {
					it("renders a heading", () => {
					  render(
						<MockedProvider mocks={mocks} addTypename={false}>
						  <Navigation />
						</MockedProvider>
					  );
				  
					  const heading = screen.getByRole("heading", {
						name: /TECTONIC HR/i,
					  });
				  
					  expect(heading).toBeInTheDocument();
					});
				  });
			\end{lstlisting}

			\noindent \underbar{\textbf{Testni Slučajevi 4, 5, 6}}
			\begin{lstlisting}
				describe("Intensity generation", () => {
					it("Expect to check simple", () => {
					  const data = ["1", "5", "5", "5", "3"];
					  expect(calcIntensity(data)).toBe(5);
					});
				  
					it("Expect to more stuff simple", () => {
					  const data = ["1", "<=3", ">=8", "7", "6", "<=4", "5", "5", "5", "3"];
					  expect(calcIntensity(data)).toBe(8);
					});
				  
					it("Expect to or simple", () => {
					  const data = ["1", "2|3"];
					  expect(calcIntensity(data)).toBe(3);
					});
				  });
			\end{lstlisting}
			
			\subsection{Ispitivanje sustava}
			
			 \textit{Ispitivanje cjelokupnog sustava provedeno je pomoću alata Selenium WebDriver. Ispitni slučajevi pokrivaju prijavu u sustav s ispravnim/neispravnim podacima, kao i uspješno/neuspješno dodavanje novog seizmologa.\\ }
			 
			 \noindent \underbar{\textbf{Testni Slučaj 1: Prijava s ispravnim podacima}}
			 \begin{packed_item}

				 \item \textbf{Ulaz:}

				 \item[] \begin{packed_enum}
					\item email: "foo@bar.com"
					\item password: "foobar123"
				\end{packed_enum}

				  \item \textbf{Očekivani izlaz:}

				 \item[] \begin{packed_enum}
					\item prijava u sustav
					\item preusmjerenje na početnu stranicu
				\end{packed_enum}
				 
				\item \textbf{Rezultat:}

				\item[] \begin{packed_enum}
					\item prijava u sustav
					\item preusmjerenje na početnu stranicu
			   \end{packed_enum}

			 \end{packed_item}
			 \begin{lstlisting}
				@Test()
				public void loginTestGoodCredentials() {
					System.setProperty("webdriver.chrome.driver", "C:\\Program Files\\chromedriver\\chromedriver.exe");
					WebDriver driver = new ChromeDriver();
					driver.manage().timeouts().implicitlyWait(10, TimeUnit.SECONDS);
					// posjeti web stranicu
					driver.get("https://tectonichr.tk/");

					// otvori izbornik s gumbom za prijavu
					WebElement element = driver.findElement(By.id("menu-button-2"));
					element.click();

					// otvori stranicu za prijavu
					element = driver.findElement(By.id("menu-list-2"));
					element.click();

					// unesi email
					element = driver.findElement(By.name("email"));
					element.sendKeys("foo@bar.com");

					// unesi zaporku
					element = driver.findElement(By.name("password"));
					element.sendKeys("foobar123");

					// pritiskom na gumb pokusaj se prijaviti u sustav
					driver.findElement(By.cssSelector("button[type='submit']")).click();

					try {
						// pricekaj dok se provjeri prijava
						TimeUnit.MILLISECONDS.sleep(1000);
					} catch (Exception e) {
						System.out.println("Exception while waiting for login: " + e.getMessage());
					}

					// ako je doslo do preusmjerenja na pocetnu stranicu prijava je uspjesna
					String redirURL = driver.getCurrentUrl();
					boolean compRes = redirURL.equals("https://tectonichr.tk/");
					assertEquals(compRes, true);

					driver.quit();
				}
			\end{lstlisting}

			\noindent \underbar{\textbf{Testni Slučaj 2: Prijava s neispravnim podacima}}
			 \begin{packed_item}

				 \item \textbf{Ulaz:}

				 \item[] \begin{packed_enum}
					\item email: "neispravni@email.com"
					\item password: "neispravnaZaporka"
				\end{packed_enum}

				  \item \textbf{Očekivani izlaz:}

				 \item[] \begin{packed_enum}
					\item neuspješna prijava u sustav
				\end{packed_enum}
				 
				\item \textbf{Rezultat:}

				\item[] \begin{packed_enum}
					\item neuspješna prijava u sustav
			   \end{packed_enum}

			 \end{packed_item}
			\begin{lstlisting}
				@Test()
				public void loginTestBadCredentials() {
					System.setProperty("webdriver.chrome.driver", "C:\\Program Files\\chromedriver\\chromedriver.exe");
					WebDriver driver = new ChromeDriver();
					driver.manage().timeouts().implicitlyWait(10, TimeUnit.SECONDS);
					// posjeti web stranicu
					driver.get("https://tectonichr.tk/");

					// otvori izbornik s gumbom za prijavu
					WebElement element = driver.findElement(By.id("menu-button-2"));
					element.click();

					// otvori stranicu za prijavu
					element = driver.findElement(By.id("menu-list-2"));
					element.click();

					// unesi email
					element = driver.findElement(By.name("email"));
					element.sendKeys("krivi@email.com");

					// unesi zaporku
					element = driver.findElement(By.name("password"));
					element.sendKeys("neispravnaSifra");

					// pritiskom na gumb pokusaj se prijaviti u sustav
					driver.findElement(By.cssSelector("button[type='submit']")).click();

					try {
						// pricekaj dok se provjeri prijava
						TimeUnit.MILLISECONDS.sleep(1000);
					} catch (Exception e) {
						System.out.println("Exception while waiting for login: " + e.getMessage());
					}

					// ako nije doslo do preusmjerenja uspjesno je prepoznata priajva s neispravnim
					// podacima
					String redirURL = driver.getCurrentUrl();
					boolean compRes = redirURL.equals("https://tectonichr.tk/auth/login");
					assertEquals(compRes, true);

					driver.quit();
				}
				\end{lstlisting}

			\noindent \underbar{\textbf{Testni Slučaj 3: Unos novog seizmologa sa ispravnom e-mail adresom}}
			 \begin{packed_item}

				 \item \textbf{Ulaz:}

				 \item[] \begin{packed_enum}
					\item email: "slučajno izgenerirana e-mail adresa"
					\item password: "123456"
				\end{packed_enum}

				  \item \textbf{Očekivani izlaz:}

				 \item[] \begin{packed_enum}
					\item novi seizmolog
					\item preusmjerenje na stranicu sa seizmolozima
				\end{packed_enum}
				 
				\item \textbf{Rezultat:}

				\item[] \begin{packed_enum}
					\item novi seizmolog
					\item preusmjerenje na stranicu sa seizmolozima
			   \end{packed_enum}

			 \end{packed_item}

			 \begin{lstlisting}
				@Test()
				public void addSeismologistValidEmail() {
					System.setProperty("webdriver.chrome.driver", "C:\\Program Files (x86)\\Chrome Driver\\chromedriver.exe");
					WebDriver driver = new ChromeDriver();
					driver.manage().timeouts().implicitlyWait(10, TimeUnit.SECONDS);

					// posjeti pocetnu stranicu
					driver.get("https://tectonichr.tk/");

					// otvori izbornik
					WebElement element = driver.findElement(By.id("menu-button-2"));
					element.click();

					// pritisni gumb i otvori stranicu za prijavu
					element = driver.findElement(By.id("menu-list-2"));
					element.click();

					// unesi email adresu
					element = driver.findElement(By.name("email"));
					element.sendKeys("foo@bar.com");

					// unesi zaporku
					element = driver.findElement(By.name("password"));
					element.sendKeys("foobar123");

					// posalji podatke i provjeri
					driver.findElement(By.cssSelector("button[type='submit']")).click();
					
					try {
						// pricekaj promjenu stranice
						TimeUnit.MILLISECONDS.sleep(1000);
					} catch (Exception e) {
						System.out.println("Exception while waiting for login: " + e.getMessage());
					}
					// otvori izbornik
					element = driver.findElement(By.id("menu-button-2"));
					element.click();

					// otvori stranicu seizmologa
					element = driver.findElement(By.id("menu-list-2-menuitem-4"));
					element.click();

					// pritisni gumb za dodavanje novog seizmologa
					element = driver.findElement(By.className("css-415j94"));
					element.click();

					// generiraj random email
					Random randomGenerator = new Random();  
					int randomInt = randomGenerator.nextInt(1000);  
					String email="username"+ randomInt +"@gmail.com";

					// unesi email
					element = driver.findElement(By.name("email"));
					element.sendKeys(email);

					// unesi zaporku
					element = driver.findElement(By.name("password"));
					element.sendKeys("123456");
					
					// posalji podatke i provjeri
					driver.findElement(By.cssSelector("button[type='submit']")).click();
					
					try {
						// pricekaj promjenu stranice
						TimeUnit.MILLISECONDS.sleep(1000);
					} catch (Exception e) {
						System.out.println("Exception while waiting for login: " + e.getMessage());
					}
					// dohvati promjenjenu stranicu
					String redirURL = driver.getCurrentUrl();

					// ako je doslo do preusmjerenja, uspjesno je dodan novi seizmolog
					boolean compRes = redirURL.equals("https://tectonichr.tk/admin/users");
					assertEquals(compRes, true);
					
					driver.quit();
				}
			 \end{lstlisting}

			 \noindent \underbar{\textbf{Testni Slučaj 4: Unos novog seizmologa sa postojećom e-mail adresom}}
			 \begin{packed_item}

				 \item \textbf{Ulaz:}

				 \item[] \begin{packed_enum}
					\item email: "existing@email.com"
					\item password: "123456"
				\end{packed_enum}

				  \item \textbf{Očekivani izlaz:}

				 \item[] \begin{packed_enum}
					\item neuspješno dodavanje novog seizmologa
					\item poruka o već postojećoj e-mail adresi
				\end{packed_enum}
				 
				\item \textbf{Rezultat:}

				\item[] \begin{packed_enum}
					\item neuspješno dodavanje novog seizmologa
					\item poruka o već postojećoj e-mail adresi
			   \end{packed_enum}

			 \end{packed_item}
			 \begin{lstlisting}
				@Test()
				public void addSeismologistExistingEmail() {
					System.setProperty("webdriver.chrome.driver", "C:\\Program Files (x86)\\Chrome Driver\\chromedriver.exe");
					WebDriver driver = new ChromeDriver();
					driver.manage().timeouts().implicitlyWait(10, TimeUnit.SECONDS);

					// posjeti pocetnu stranicu
					driver.get("https://tectonichr.tk/");

					// otvori izbornik
					WebElement element = driver.findElement(By.id("menu-button-2"));
					element.click();

					// pritisni gumb i otvori stranicu za prijavu
					element = driver.findElement(By.id("menu-list-2"));
					element.click();

					// unesi email
					element = driver.findElement(By.name("email"));
					element.sendKeys("foo@bar.com");

					// unesi zaporku
					element = driver.findElement(By.name("password"));
					element.sendKeys("foobar123");

					// posalji podatke i provjeri
					driver.findElement(By.cssSelector("button[type='submit']")).click();
					
					try {
						// pricekaj promjenu stranice
						TimeUnit.MILLISECONDS.sleep(1000);
					} catch (Exception e) {
						System.out.println("Exception while waiting for login: " + e.getMessage());
					}
					// otvori izbornik
					element = driver.findElement(By.id("menu-button-2"));
					element.click();

					// otvori stranicu seizmologa
					element = driver.findElement(By.id("menu-list-2-menuitem-4"));
					element.click();

					// pritisni gumb za dodavanje novog seizmologa
					element = driver.findElement(By.className("css-415j94"));
					element.click();

					// unesi postojeci email 
					element = driver.findElement(By.name("email"));
					element.sendKeys("existing@email.com");

					// unesi zaporku
					element = driver.findElement(By.name("password"));
					element.sendKeys("123456");
					
					// posalji podatke i provjeri
					driver.findElement(By.cssSelector("button[type='submit']")).click();
					
					try {
						// pricekaj promjenu stranice
						TimeUnit.MILLISECONDS.sleep(1000);
					} catch (Exception e) {
						System.out.println("Exception while waiting for login: " + e.getMessage());
					}
					
					String redirURL = driver.getCurrentUrl();

					// ako stranica nije promjenjena, email vec postoji
					boolean compRes = redirURL.equals("https://tectonichr.tk/admin/users/new");
					assertEquals(compRes, true);
					
					driver.quit();	
				}
			 \end{lstlisting}

			
			\eject 
		
		
		\section{Dijagram razmještaja}
			
			\textbf{\textit{dio 2. revizije}}
			
			 \textit{Potrebno je umetnuti \textbf{specifikacijski} dijagram razmještaja i opisati ga. Moguće je umjesto specifikacijskog dijagrama razmještaja umetnuti dijagram razmještaja instanci, pod uvjetom da taj dijagram bolje opisuje neki važniji dio sustava.}
			
			\eject 
		
		\section{Upute za puštanje u pogon}
		
			\textbf{\textit{dio 2. revizije}}\\
		
			 \textit{U ovom poglavlju potrebno je dati upute za puštanje u pogon (engl. deployment) ostvarene aplikacije. Na primjer, za web aplikacije, opisati postupak kojim se od izvornog kôda dolazi do potpuno postavljene baze podataka i poslužitelja koji odgovara na upite korisnika. Za mobilnu aplikaciju, postupak kojim se aplikacija izgradi, te postavi na neku od trgovina. Za stolnu (engl. desktop) aplikaciju, postupak kojim se aplikacija instalira na računalo. Ukoliko mobilne i stolne aplikacije komuniciraju s poslužiteljem i/ili bazom podataka, opisati i postupak njihovog postavljanja. Pri izradi uputa preporučuje se \textbf{naglasiti korake instalacije uporabom natuknica} te koristiti što je više moguće \textbf{slike ekrana} (engl. screenshots) kako bi upute bile jasne i jednostavne za slijediti.}
			
			
			 \textit{Dovršenu aplikaciju potrebno je pokrenuti na javno dostupnom poslužitelju. Studentima se preporuča korištenje neke od sljedećih besplatnih usluga: \href{https://aws.amazon.com/}{Amazon AWS}, \href{https://azure.microsoft.com/en-us/}{Microsoft Azure} ili \href{https://www.heroku.com/}{Heroku}. Mobilne aplikacije trebaju biti objavljene na F-Droid, Google Play ili Amazon App trgovini.}
			
			
			\eject 