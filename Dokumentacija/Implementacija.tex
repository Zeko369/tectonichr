\chapter{Implementacija i korisničko sučelje}
		
		
		\section{Korištene tehnologije i alati}
		Pisana komunikacija u timu odvijala se putem aplikacije WhatsApp\footnote{https://www.whatsapp.com/}, a sastanci putem Google Meeta\footnote{https://meet.google.com/} i Discorda\footnote{https://discord.com/}. WhatsApp je besplatna mobilna aplikacija koja služi za razmjenu poruka, fotografija i videozapisa putem mobilnog interneta pametnim telefonom. Google Meet je usluga videokomunikacije koju je razvio Google, dok je Discord VoIP aplikacija koja omogućava komunikaciju glasom, videom i tekstom.
		
		UML dijagrami napravljeni su alatom Visual Paradigm Online\footnote{https://online.visual-paradigm.com/}. Visual Paradigm Online je online alat za crtanje dijagrama i njihovu pohranu u web-pregledniku koji omogućava istovremeni rad više korisnika u stvarnom vremenu. Dijagrame možemo preuzeti u raznim formatima (.png, .jpeg, .pdf i dr.).
		
		Za vizualizaciju stranice korištena je Figma\footnote{https://www.figma.com/}, besplatan alat za UI i UX dizajn, uređivanje i izradu	prototipova te generiranje koda dostupan na webu ili u obliku desktop aplikacije.
		
		Izvornim kodom upravljano je sustavom Git\footnote{https://git-scm.com/}. Git je distribuirani sustav za upravljanje različitim	verzijama podataka (programskog koda, teksta i dr.). Sastoji se od udaljenog repozitorija koji se nalazi na nekoj Git platformi u oblaku i od lokalnih kopija tog repozitorija na računalima korisnika koji rade na projektu. Udaljeni repozitorij ovog projekta dostupan je na web platformi Gitlab\footnote{https://about.gitlab.com/}.
		
		Kao razvojno okruženje korišten je Visual Studio Code\footnote{https://code.visualstudio.com/}. Visual Studio Code
		je uređivač izvornog koda koji je razvio Microsoft za Linux, Windows i Mac OS platforme. Uključuje podršku za uklanjanje pogrešaka, isticanje sintakse, inteligentno dovršavanje koda, isječke, refaktoriranje koda i ugrađeni Git.
		
		Osim VSCode-a, koristili smo i JetBrains WebStorm\footnote{https://www.jetbrains.com/webstorm/} i JetBrains DataGrip\footnote{https://www.jetbrains.com/datagrip/}. WebStorm je integrirana razvojna okolina za JavaScript i povezane tehnologije, a DataGrip detektira moguće \textit{buggove} u kodu i predlaže najbolje opcije za njihovo ispravljanje.
		
		Cijeli sustav pisan je jezikom TypeScript\footnote{https://www.typescriptlang.org/} koji je proširenje jezika JavaScript(JavaScript s tipovima), skriptnog programskog jezika koji omogućava interakciju korisnika s web-stranicom.
		
		Za frontend smo koristili React\footnote{https://reactjs.org/} i Chakra UI\footnote{https://chakra-ui.com/}. React je knjižnica koja služi za izgradnju korisničkog sučelja ili UI komponenti, a Chakra je jednostavna modularna bibloteka komponenata	u kojoj se nalaze blokovi potrebni za izgradnju React aplikacija.
		
		 PostgreSQL\footnote{https://www.postgresql.org/} baza podataka spremljena je na $Heroku^{14}$	koji je ujedno i server. PostgreSQL je besplatni i relacijski sustav upravljanja bazom podataka otvorenog koda dizajniran za upravljanje nizom radnih opterećenja, od pojedinačnih strojeva do skladišta podataka ili web-usluga s mnogim istovremenim korisnicima. Heroku je platforma u oblaku, točnije platforma kao usluga (engl. Platform as a Service, skraćeno PaaS), što znači da korisnici	na nju postavljaju ili na njoj izraduju aplikaciju koja će se na njoj i izvršavati.
		
			\eject 
		
	
		\section{Ispitivanje programskog rješenja}
			
			\textbf{\textit{dio 2. revizije}}\\
			
			 \textit{U ovom poglavlju je potrebno opisati provedbu ispitivanja implementiranih funkcionalnosti na razini komponenti i na razini cijelog sustava s prikazom odabranih ispitnih slučajeva. Studenti trebaju ispitati temeljnu funkcionalnost i rubne uvjete.}
	
			
			\subsection{Ispitivanje komponenti}
			\textit{Potrebno je provesti ispitivanje jedinica (engl. unit testing) nad razredima koji implementiraju temeljne funkcionalnosti. Razraditi \textbf{minimalno 6 ispitnih slučajeva} u kojima će se ispitati redovni slučajevi, rubni uvjeti te izazivanje pogreške (engl. exception throwing). Poželjno je stvoriti i ispitni slučaj koji koristi funkcionalnosti koje nisu implementirane. Potrebno je priložiti izvorni kôd svih ispitnih slučajeva te prikaz rezultata izvođenja ispita u razvojnom okruženju (prolaz/pad ispita). }
			
			
			
			\subsection{Ispitivanje sustava}
			
			 \textit{Potrebno je provesti i opisati ispitivanje sustava koristeći radni okvir Selenium\footnote{\url{https://www.seleniumhq.org/}}. Razraditi \textbf{minimalno 4 ispitna slučaja} u kojima će se ispitati redovni slučajevi, rubni uvjeti te poziv funkcionalnosti koja nije implementirana/izaziva pogrešku kako bi se vidjelo na koji način sustav reagira kada nešto nije u potpunosti ostvareno. Ispitni slučaj se treba sastojati od ulaza (npr. korisničko ime i lozinka), očekivanog izlaza ili rezultata, koraka ispitivanja i dobivenog izlaza ili rezultata.\\ }
			 
			 \textit{Izradu ispitnih slučajeva pomoću radnog okvira Selenium moguće je provesti pomoću jednog od sljedeća dva alata:}
			 \begin{itemize}
			 	\item \textit{dodatak za preglednik \textbf{Selenium IDE} - snimanje korisnikovih akcija radi automatskog ponavljanja ispita	}
			 	\item \textit{\textbf{Selenium WebDriver} - podrška za pisanje ispita u jezicima Java, C\#, PHP koristeći posebno programsko sučelje.}
			 \end{itemize}
		 	\textit{Detalji o korištenju alata Selenium bit će prikazani na posebnom predavanju tijekom semestra.}
			
			\eject 
		
		
		\section{Dijagram razmještaja}
			
			\textbf{\textit{dio 2. revizije}}
			
			 \textit{Potrebno je umetnuti \textbf{specifikacijski} dijagram razmještaja i opisati ga. Moguće je umjesto specifikacijskog dijagrama razmještaja umetnuti dijagram razmještaja instanci, pod uvjetom da taj dijagram bolje opisuje neki važniji dio sustava.}
			
			\eject 
		
		\section{Upute za puštanje u pogon}
		
			\textbf{\textit{dio 2. revizije}}\\
		
			 \textit{U ovom poglavlju potrebno je dati upute za puštanje u pogon (engl. deployment) ostvarene aplikacije. Na primjer, za web aplikacije, opisati postupak kojim se od izvornog kôda dolazi do potpuno postavljene baze podataka i poslužitelja koji odgovara na upite korisnika. Za mobilnu aplikaciju, postupak kojim se aplikacija izgradi, te postavi na neku od trgovina. Za stolnu (engl. desktop) aplikaciju, postupak kojim se aplikacija instalira na računalo. Ukoliko mobilne i stolne aplikacije komuniciraju s poslužiteljem i/ili bazom podataka, opisati i postupak njihovog postavljanja. Pri izradi uputa preporučuje se \textbf{naglasiti korake instalacije uporabom natuknica} te koristiti što je više moguće \textbf{slike ekrana} (engl. screenshots) kako bi upute bile jasne i jednostavne za slijediti.}
			
			
			 \textit{Dovršenu aplikaciju potrebno je pokrenuti na javno dostupnom poslužitelju. Studentima se preporuča korištenje neke od sljedećih besplatnih usluga: \href{https://aws.amazon.com/}{Amazon AWS}, \href{https://azure.microsoft.com/en-us/}{Microsoft Azure} ili \href{https://www.heroku.com/}{Heroku}. Mobilne aplikacije trebaju biti objavljene na F-Droid, Google Play ili Amazon App trgovini.}
			
			
			\eject 