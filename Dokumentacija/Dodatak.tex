\chapter*{Dodatak: Prikaz aktivnosti grupe}
		\addcontentsline{toc}{chapter}{Dodatak: Prikaz aktivnosti grupe}
		
		\section*{Dnevnik sastajanja}
		
		\textbf{\textit{Kontinuirano osvježavanje}}\\
		
		 \textit{U ovom dijelu potrebno je redovito osvježavati dnevnik sastajanja prema predlošku.}
		
		\begin{packed_enum}
			\item  sastanak
			
			\item[] \begin{packed_item}
				\item Datum: 9.10.2021.
				\item Prisustvovali: M.Ćurković, D.Čemeljić, A.Engler, K.Iličić, D.Vorkapić
				\item Teme sastanka:
				\begin{packed_item}
					\item  upoznavanje članova tima
				\end{packed_item}
			\end{packed_item}
			
			\item  sastanak
			\item[] \begin{packed_item}
				\item Datum: 19.10.2021.
				\item Prisustvovali: M.Ćurković, D.Čemeljić, A.Engler, K.Iličić, M.Jurić, D.Vorkapić, F.Zekan
				\item Teme sastanka:
				\begin{packed_item}
					\item  rasprava o nedoumicama u zadanom zadatku
					\item  razjašnjavanje zahtjeva
					\item  dogovor o korištenim tehnologijama
					\item  upoznavanje s korištenim tehnologijama
					\item  razrada specifikacije programske potpore
				\end{packed_item}
			\end{packed_item}
		
			\item  sastanak
			\item[] \begin{packed_item}
				\item Datum: 26.10.2021.
				\item Prisustvovali: M.Ćurković, D.Čemeljić, A.Engler, K.Iličić, M.Jurić, D.Vorkapić
				\item Teme sastanka:
				\begin{packed_item}
					\item  pregled dosad izrađene dokumentacije
					\item  prijedlog potrebnih izmjena
					\item  dogovor o izgledu stranice
					\item  sastavljanje prototipa u alatu Figma
					\item  podjela zadataka
				\end{packed_item}
			\end{packed_item}
		
			\item  sastanak
			\item[] \begin{packed_item}
				\item Datum: 9.11.20221.
				\item Prisustvovali: M.Ćurković, D.Čemeljić, A.Engler, K.Iličić, M.Jurić, D.Vorkapić, F.Zekan
				\item Teme sastanka:
				\begin{packed_item}
					\item  pregled dosad izrađene dokumentacije
					\item  diskusija o implementaciji aplikacije
				\end{packed_item}
			\end{packed_item}
			
			%
			
		\end{packed_enum}
		
		\eject
		\section*{Tablica aktivnosti}
		
			\textbf{\textit{Kontinuirano osvježavanje}}\\
			
			 \textit{Napomena: Doprinose u aktivnostima treba navesti u satima po članovima grupe po aktivnosti.}

			\begin{longtblr}[
					label=none,
				]{
					vlines,hlines,
					width = \textwidth,
					colspec={X[7, l]X[1, c]X[1, c]X[1, c]X[1, c]X[1, c]X[1, c]X[1, c]}, 
					vline{1} = {1}{text=\clap{}},
					hline{1} = {1}{text=\clap{}},
					rowhead = 1,
				} 
				\multicolumn{1}{c|}{} & \multicolumn{1}{c|}{\rotatebox{90}{\textbf{Maja Jurić }}} & \multicolumn{1}{c|}{\rotatebox{90}{\textbf{David Čemeljić }}} &	\multicolumn{1}{c|}{\rotatebox{90}{\textbf{Mihovil Ćurković }}} & \multicolumn{1}{c|}{\rotatebox{90}{\textbf{Antonija Engler }}} &	\multicolumn{1}{c|}{\rotatebox{90}{\textbf{Klara Iličić }}} & \multicolumn{1}{c|}{\rotatebox{90}{\textbf{Dalijo Vorkapić }}} &	\multicolumn{1}{c|}{\rotatebox{90}{\textbf{Fran Zekan }}} \\  
				Upravljanje projektom 		&  9  &  &  &  &  &  & \\ 
				Opis projektnog zadatka 	&  3  &  &  &  &  4  & \\ 
				
				Funkcionalni zahtjevi       &  &  2  &  3  &  1  &  & \\ 
				Opis pojedinih obrazaca 	&  3  &  &  &  3  &  3 \\ 
				Dijagram obrazaca 			&  3  &  &  &  &  3  & \\
				Sekvencijski dijagrami 		&  4  &  &  &  &  4  &\\ 
				Opis ostalih zahtjeva 		&  &  2  &  &  &  &  \\ 

				Arhitektura i dizajn sustava	 &  & 4 &  &  &  &  3  & 4 \\ 
				Baza podataka				&  &  &  &  &  & 3 & 3  \\ 
				Dijagram razreda 			&  &  &  &  &  &  & 1 \\ 
				Dijagram stanja				&  &  &  &  &  &  &  \\ 
				Dijagram aktivnosti 		&  &  &  &  &  &  &  \\ 
				Dijagram komponenti			&  &  &  &  &  &  &  \\ 
				Korištene tehnologije i alati 		&  &  &  &  &  &  &  \\ 
				Ispitivanje programskog rješenja 	&  &  &  &  &  &  &  \\ 
				Dijagram razmještaja			&  &  &  &  &  &  &  \\ 
				Upute za puštanje u pogon 		&  &  &  &  &  &  &  \\  
				Dnevnik sastajanja 			&  &  &  &  &  &  &  \\ 
				Zaključak i budući rad 		&  &  &  &  &  &  &  \\  
				Popis literature 			&  &  &  &  &  &  & 0.5 \\  
				&  &  &  &  &  &  &  \\ \hline 
				Inicjalizacija aplikacije 	&  &  &  &  &  &  & 4 \\
				Autentikacija 				&  & 3 &  &  &  &  &  \\ 
				Upitnici 					&  &  & 3 &  &  &  &  \\ 
				Kreiranje potresa 			&  &  &  &  &  & 3 &  \\ 					
				Puštanje u pogon 			&  &  &  &  &  &  & 2 \\  
				 							&  &  &  &  &  &  &\\ 
			\end{longtblr}
					
					
		\eject
		\section*{Dijagrami pregleda promjena}
		
		\textbf{\textit{dio 2. revizije}}\\
		
		\textit{Prenijeti dijagram pregleda promjena nad datotekama projekta. Potrebno je na kraju projekta generirane grafove s gitlaba prenijeti u ovo poglavlje dokumentacije. Dijagrami za vlastiti projekt se mogu preuzeti s gitlab.com stranice, u izborniku Repository, pritiskom na stavku Contributors.}
		
	