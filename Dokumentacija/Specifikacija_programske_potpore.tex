\chapter{Specifikacija programske potpore}
		
	\section{Funkcionalni zahtjevi}
			
			\noindent \textbf{Dionici:}
			
			\begin{packed_enum}
				
				\item Neregistrirani/neprijavljeni korisnik (građanin)
				\item Administrator				
				\item Seizmolog/znanstvenik
				
			\end{packed_enum}
			
			\noindent \textbf{Aktori i njihovi funkcionalni zahtjevi:}
			
			
			\begin{packed_enum}
				\item  \underbar{Neregistrirani/neprijavljeni korisnik (građanin) može:}
				
				\begin{packed_enum}
					
					\item ispuniti upitnik ako je osjetio novi potres
					\item pregledati sve arhivirane potrese s:
					\begin{packed_enum}

						\item interaktivnom kartom Hrvatske 
						\item dodatnim informacijama (broj ispunjenih upitnika, datum, vrijeme, geografski položaj, dubina fokusa, magnituda i naziv područja)
					
					\end{packed_enum}
					\item pristupiti aktualnim potresima i:
					\begin{packed_enum}

						\item ispuniti upitnik
						\item pregledati već analizirane podatke (karta, broj ispunjenih upitnika, datum, vrijeme, itd.)
					
					\end{packed_enum}
				\end{packed_enum}
			
				\item  \underbar{Administrator može:}
				
				\begin{packed_enum}

					\item sve što i neregistrirani/neprijavljeni korisnik može
					\item stvoriti novi potres od 10 ili više upitnika
					\begin{packed_enum}

						\item dodati upitnik nekom od postojećih potresa
					
					\end{packed_enum}
					\item registrirati nove seizmologe/znanstvenike slanjem e-mail pozivnice
					\item poslati notifikaciju korisnicima da ispune upitnik za mogući potres koji se upravo dogodio
					\item dodavati, mijenjati i brisati podatke iz baze podataka
					
				\end{packed_enum}

				\item  \underbar{Seizmolog/znanstvenik može:}

				\begin{packed_enum}

					\item pristupiti aplikaciji nakon registracije (e-mail i lozinka)
					\item pregledavati:
					\begin{packed_enum}
						
						\item baze prikupljenih podataka
						\item preliminarnu kartu intenziteta

					\end{packed_enum}
					\item u tekstualnom formatu preuzeti:
					\begin{packed_enum}

						\item odgovore na upitnik
						\item srednje vrijednosti/položaje intenziteta na karti

					\end{packed_enum}
				\end{packed_enum}
			\end{packed_enum}
			
			\eject 
			
			
				
			\subsection{Obrasci uporabe}
				
				\textbf{\textit{dio 1. revizije}}
				
				\subsubsection{Opis obrazaca uporabe}
					\textit{Funkcionalne zahtjeve razraditi u obliku obrazaca uporabe. Svaki obrazac je potrebno razraditi prema donjem predlošku. Ukoliko u nekom koraku može doći do odstupanja, potrebno je to odstupanje opisati i po mogućnosti ponuditi rješenje kojim bi se tijek obrasca vratio na osnovni tijek.}\\
					

					\noindent \underbar{\textbf{UC$<$broj obrasca$>$ -$<$ime obrasca$>$}}
					\begin{packed_item}
	
						\item \textbf{Glavni sudionik: }$<$sudionik$>$
						\item  \textbf{Cilj:} $<$cilj$>$
						\item  \textbf{Sudionici:} $<$sudionici$>$
						\item  \textbf{Preduvjet:} $<$preduvjet$>$
						\item  \textbf{Opis osnovnog tijeka:}
						
						\item[] \begin{packed_enum}
	
							\item $<$opis korak jedan$>$
							\item $<$opis korak dva$>$
							\item $<$opis korak tri$>$
							\item $<$opis korak četiri$>$
							\item $<$opis korak pet$>$
						\end{packed_enum}
						
						\item  \textbf{Opis mogućih odstupanja:}
						
						\item[] \begin{packed_item}
	
							\item[2.a] $<$opis mogućeg scenarija odstupanja u koraku 2$>$
							\item[] \begin{packed_enum}
								
								\item $<$opis rješenja mogućeg scenarija korak 1$>$
								\item $<$opis rješenja mogućeg scenarija korak 2$>$
								
							\end{packed_enum}
							\item[2.b] $<$opis mogućeg scenarija odstupanja u koraku 2$>$
							\item[3.a] $<$opis mogućeg scenarija odstupanja  u koraku 3$>$
							
						\end{packed_item}
					\end{packed_item}
				
					
				\subsubsection{Dijagrami obrazaca uporabe}
					
					\textit{Prikazati odnos aktora i obrazaca uporabe odgovarajućim UML dijagramom. Nije nužno nacrtati sve na jednom dijagramu. Modelirati po razinama apstrakcije i skupovima srodnih funkcionalnosti.}
				\eject		
				
			\subsection{Sekvencijski dijagrami}
				
				\textbf{\textit{dio 1. revizije}}\\
				
				\textit{Nacrtati sekvencijske dijagrame koji modeliraju najvažnije dijelove sustava (max. 4 dijagrama). Ukoliko postoji nedoumica oko odabira, razjasniti s asistentom. Uz svaki dijagram napisati detaljni opis dijagrama.}
				\eject
	
		\section{Ostali zahtjevi}
				\begin{packed_item}
					\item Aplikacija treba podržavati rad više korisnika u stvarnom vremenu.
					\item Sustav treba biti implementiran kao mobilna ili web aplikacija koristeći objektno orijentirane jezike.
					\item Neposredno nakon što se dogodi neki potres, administrator može poslati notifikaciju na uređaje građana da ispune upitnik.
					\item Treba kreirati administratora i dva znanstvenika te barem jedan stari potres s 10 popunjenih upitnika i jedan novi potres s 5 popunjenih upitnika.
					\item Korisničko sučelje mora biti pisano na hrvatskom jeziku i podržavati dijakritičke znakove.
					\item Upitnik o potresu mora imati barem 10 pitanja pomoću kojih se može odrediti intenzitet potresa.
					\item Aplikacija mora jednostavno prikazivati informacije o potresima korisnicima.
					\item Karta treba prikazivati sve gradove i mjesta u kojima su upitnici popunjeni.
					\item Karta treba prikazivati podatke samo s jednog potresa, ne svih koji su se do sad dogodili.
				\end{packed_item}