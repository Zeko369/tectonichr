\chapter{Opis projektnog zadatka}
		


{Ovaj se projekt bavi razvojem programske podrške za web aplikaciju "TectonicHR". 
	Cilj aplikacije "TectonicHR" olakšano je prikupljanje podataka o intenzitetu potres te olakšani vizualni pristup informacijama. 
	Aplikacija je namijenjena znanstvenoj zajednici, ali i općoj populaciji.
	 
	Znanstvenoj zajednici, seizmolozima, bit će olakšan pristup informacijama i njihovo prikupljanje.  
	Opća populacija, građani, imat će mogućnost unosa novog potresa kojega su osjetili te pregled već zabilježenih potresa.  
	
	Prilikom otvaranja aplikacije građanima se prikazuje početna stranica na kojoj se nude tri opcije:
		\begin{packed_item}
			\item Novi upitnik?
			\item Aktualni potresi
			\item Arhivirani potresi
		\end{packed_item}
	Na početnoj stranici također se nalazi i preliminarna karta Hrvatske na kojoj su označeni aktualni i arhivirani potresi. Boje oznaka potresa različite su za ove dvije kategorije potresa.
	
	Odabirom opcije „Aktualni potresi“ prikazuje se stranica na kojoj se može pregledati karta aktualnih potresa, mogu se pregledati podatci o aktualnim potresima te se može ispuniti upitnik za neki od navedenih aktualnih potresa.
	
	Odabirom opcije „Arhivirani potresi“ prikazuje se stranica na kojoj se može pregledati karta intenziteta arhiviranih potresa i na kojoj se mogu pregledati podatci o arhiviranim potresima.
	
	Na kartama na stranicama „Aktualni potresi“ i „Arhivirani potresi“ zvjezdicom je označen epicentar te kružićem određene boje označen je intenzitet potresa. Hladnije plave nijanse označavaju slabiji intezitet, a tamnije crvene jači intezitet. 
	Pored svake od karta nalazi se tablica s podacima o zadnjem i starijim potresima.
	
	Pri pregledu arhiviranih i aktualnih potresa moguće je filtrirati potrese prema mjestu, vremenu ili intezitetu. 
	
	Ako se posljednji prijavljeni potres ne poklapa s opažanjima građana, preko treće opcije „Novi potres?" građanin može prijaviti novi potres koji je osjetio. Građanin prijavljuje novi potres ispunjavanjem i predajom upitnika.
	
	Upitnik se sastoji od pitanja iz kojih znanstvenici mogu dobiti vrijedne informacije o tome kakvi su bili učinci potresa. 
	Pitanja se odnose na to koliko se potres osjetio, koliku je štetu napravio na malim predmetima, kućama, zgradama i zemlji te kako su ljudi reagirali. 
	Odgovori na ta pitanja pomažu pri računanju intenziteta potresa.
	
	Na početnoj stranici u gornjem desnom kutu nalazi se gumb za prijavu u sustav. Tu funkcionalnost koriste seizmolozi koji prijavom mogu preuzeti podatke o potresu i odgovore građana na upitnik.  Pri prijavi upisuju svoje e-mail i lozinku.
	 
	Znanstvenike u sustav mora registrirati administrator. Pri registraciji, administrator upisuje:
	\begin{packed_item}
		\item ime seizmologa
		\item prezime seizmologa
		\item e-mail seizmologa
		\item lozinku.
	\end{packed_item}
	Nakon što administrator registrira seizmologa u sustav, seizmolog će e-mailom primiti obavijest o registraciji te time dobiva mogućnost prijave u sustav.
	Pregled svih registriranih seizmologa omogućen je samo administratoru.
	Administrator se također mora prijaviti pri dolasku na stranicu kako bi imao sve ovlasti. \\
	
	}
	
	Program treba sam računati intenzitet potresa pomoću prethodno preuzetih upitnika. Prema odgovorima iz upitnika treba se automatski izračunati intenzitet potresa. 
	Vrijednost intenziteta na pojedinoj lokaciji odgovara srednjoj vrijednosti intenziteta svih upitnika ispunjenih za potres na istoj lokaciji. Mjesto epicentra aproksimira se koristeći lokacije jednakog intenziteta, a intenzitet potresa se u epicentru (predstavljen bojom) određuje koristeći \textit{Koevesligethyjevu jednadžbu}:
\begin{equation}
  I_{0} = I_{max} + 3\log\frac{r}{h} + 3\mu\alpha(r-h)
\end{equation}

\begin{itemize}                                                             
    \item $I_{max}$ - procijenjeni intenzitet potresa na udaljenosti r od hipocentra
    \item h = 10 km 
    \item $\mu=0.4343$
    \item $\alpha=0.005 km^{-1}$
    
\end{itemize} 


\section{Vrste korisnika}
Postoje tri vrste korisnika, a to su:
\begin{packed_item}
	\item anonimni korisnik (građanin)
	\item znanstvenik (seizmolog)
	\item administrator
\end{packed_item}

\underbar{Neregistriranom (anonimnom) korisniku} otvaranjem aplikacije prikazuje se izbornik u kojem može odabrati želi li pregledati aktualne potrese („Aktualni potresi“), pregledati arhivirane potrese („Arhivirani potresi“) ili ispuniti upitnik ako je osjetio novi potres („Novi potres?“). Klikom na „Aktualni potresi“ prikazuju mu se karta i popis potresa koje administrator još nije arhivirao. Anonimni korisnik može pretraživati te potrese i odabrati jedan potresa te za njega ispuniti upitnik. Na početnoj stranici, klikom na „Arhivirani potresi“ pokaže mu se karta i ispod nje popis potresa koje je administrator arhivirao. Prelaženjem kursorom preko oznake kojom je označen taj potres, prikazuju se osnovne informacije o potresu (datum, vrijeme, lokacija i intenzitet potresa).

\underbar{Seizmolog (znanstvenik)} se prijavljuje e-mailom i lozinkom. Ima sve mogućnosti kao i anonimni korisnik (ispunjavanje upitnika za novi potres, pregled karte i drugih podataka o aktualnim i arhiviranim potresima) uz još dodatnu ovlast preuzimanja podataka o potresima u .csv formatu. Seizmolog na svom profilu može promijeniti svoje podatke i lozinku.

\underbar{Administrator} ima najveće ovlasti. Početna stranica izgleda mu isto kao i seizmologu, ali mu se na stranici profila ispod njegovih podataka nalaze i dva gumba, „Novi upitnici“ i „Dodaj novog seizmologa“. Klikom na „Novi upitnici“ pregledava ispunjene upitnike koje još nije svrstao u nijedan potres. Upitnike može pridijeliti nekom već imenovanom potresu ili može stvoriti, imenovati i potvrditi novi potres te ih pridijeliti tom novostvorenom potresu. Klikom na „Dodaj novog seizmologa“ otvara mu se obrazac za registraciju novog seizmologa. Mora upisati ime, prezime, e-mail i lozinku novog seizmologa. Administrator, kao i seizmolog, može promijeniti svoje podatke. S početne stranice može pristupiti stranici aktualnih potresa. Na toj stranici može odabrati jedan ili više potres te ih arhivirati. Na stranici arhiviranih i na stranici aktualnih potresa može pregledavati podatke o potresima te također, kao i seizmolog, preuzeti podatke .csv formatu.

Sustav treba podržavati rad više korisnika u stvarnom vremenu.\\



\section{Usporedba s već postojećim rješenjima}

{Od sličnih aplikacija koje već postoje, najpoznatija je EMSC. Aplikaciju je razvila organizacija European Mediterranean Seismological Centre.
	
Početna stranica sastoji se od karte Europe i Mediterana. Na karti možete birati želite li pregledati potrese u zadnjih sat  vremena, 24 sata, 48 sati, tjedan dana ili dva tjedna. Osim karte Europe i Mediterana, postoji mogućnost otvaranja karte cijelog svijeta.
Ispod karte nalazi se tablica s informacijama o potresima koji su označeni na karti.

Osim toga, stranica nudi funkcionalnosti ispunjavanja upitnika o doživljaju potresa i dodavanje slika koje prikazuju posljedice potresa.

Postoji posebni odjeljak za seizmologe, kao i odjeljci namijenjeni projektima organizacije i publikaciji.

Problem stranice je što je prilično neintuitivna za korištenje i zastarjelog dizajna. Nudi mnogo mogućnosti u raznim izbornicima te to stvara problem nesnalaženja. Većinu građana koji posjećuju tu stranicu radi prijave potresa ili traženja informacija o potresu kojega su možda osjetili zasigurno neće zanimati projekti EMSC-a ili znanstvena publikacija.}

\begin{figure}[H]
			\includegraphics[width=\textwidth]{slike/emsc1.PNG} %veličina u odnosu na širinu linije
			\caption{Slika 2.2: početna stranica EMSC-a}
			\label{fig:promjene2} %label mora biti drugaciji za svaku sliku
		\end{figure}

{Prednost aplikacije „TectonicHR“ bila bi preglednost i mogućnost lakšeg snalaženja na karti. 
Mogućnost prijave seizmologa na stranicu omogućilo bi bolju prilagodbu stranice. Građanima pri korištenju ne bi smetali izbornici s mogućnostima koje oni ne bi koristili, već bi im se pregledno prikazivale funkcionalnosti ispunjavanja upitnika te pregleda aktualnih i starijih potresa. }

{Interes za korištenjem aplikacije imat će i građani i znanstvena zajednica. Potresi uvelike utječu na psihičko stanje ljudi. 
Kada dođe do potresa, većina želi znati gdje je epicentar, kolike je magnitude potres bio, kakvu je štetu prouzročilo u ostalim dijelovima pogođenog područja… 
Tako da bi ova jednostavna aplikacija ljudima pružila, za početak, osnovne informacije o doživljenom potresu te odgovor na neka od pitanja koja im se nameću nakon doživljenog potresa. 
Osim toga, korist od aplikacije imaju i znanstvenici koji putem opažanja građana mogu doći do vrlo vrijednih informacija. }\\		

\section{Mogućnosti nadogradnje}
{Buduće verzije aplikacije mogle bi uključivati mogućnosti:
\begin{packed_enum}
	\item umetanje slika područja pogođenih potresom
	\item pregledavanje pojedinog potresa (informacija o potresu, galerije slika...)
	\item prikaz područja s većim rizikom pojave potresa
	\item donacije za obnovu područja pogođenih potresom
	\item preuzimanje karte s označenim potresima u tom trenutku
	\item pregled potresa iz određenog vremenskog perioda ili s geografskog područja
\end{packed_enum}
Vidimo veliku korist i primjenjivost aplikacije te bi se nove funkcionalnosti dodavale nakon dobivanja povratne informacije od svih korisnika (i građana i seizmologa) u ovisnosti o njihovim potrebama.
}

	
\eject
	
	