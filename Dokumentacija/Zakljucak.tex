\chapter{Zaključak i budući rad}
		
		Naš tim radio je na razvoju web aplikacije TectonicHR koja služi olakšanom prikupljanju i prikazu podataka o potresima. Korisnike koji pristupaju aplikaciji podijelili smo u tri kategorije: administrator, građani (neregistrirani korisnici) i seizmolozi (korisnici koje registrira administrator). Potrese smo podijelili na arhivirane i aktualne. Aktualni potresi su oni za koje se još mogu ispuniti upitnici. Kada administrator arhivira neki potres, on prelazi u arhivirane potrese te se za njega više ne može ispuniti upitnik. Korisnici mogu ispuniti upitnik i za neki potres koji još nije ni u arhiviranim ni u aktualnim potresima. Na taj način oni prijavljuju novi potres. Na temelju odgovora iz upitnika računa se intenzitet potresa te se potresi prikazuju na karti bojom koja ovisi o njihovom intenzitetu.
		
		Prije početka razvoja aplikacije, tim se morao sastati i razjasniti temu zadatka nakon čega je slijedila okvirna podjela poslova (tko bi htio raditi koji dio). Prvi dio rada tima više se fokusirao na dokumentaciju projekta. Upravo ta detaljna analiza zahtjeva aplikacije kasnije je olakšala izradu same aplikacije. Ova faza rada uključivala je dokumentiranje funkcionalnih i nefunkcionalnih zahtjeva, razradu obrazaca uporabe te crtanje dijagrama (dijagram obrazaca uporabe, sekvencijski dijagram, dijagram razreda, model baze podataka). Za izradu dijagrama i dokumentiranje projekta bilo je potrebno znanje s predavanja predmeta te se svaki dio dokumentacije pisao nakon što bi bio ispričan na predavanju. U prvom dijelu rada na projektu također smo izradili prototip u Figmi koji nam je dao ideju izgleda konačne verzije aplikacije i olakšao u izradi frontenda.
		
		Druga faza projekta više se orijentirala na pisanje koda. Za izradu frontenda se koristio React što je većini članova bilo nepoznato te je zahtijevalo više samostalnog i ubrzanog
		učenja. Obrasci i dijagrami izrađeni u prvoj fazi su uvelike pomogli u izradi frontenda i backenda jer su
		pokazali kako bi aplikacija trebala komunicirati s korisnikom te odrediti gdje smije pristupiti, a gdje ne jer nemaju svi korisnici ista prava i mogućnosti u aplikaciji. U ovoj fazi smo dokumentirali dijagram stanja, dijagram aktivnosti, dijagram razmještaja i dijagram komponenti.
		
		Dogovori među članovima tima su se odvijali preko Whatsappa te na čestim sastancima uživo čime smo postigli informiranost svih članova grupe o napretku projekta. Sudjelovanje na ovakvom projektu bilo je vrijedno iskustvo svim članovima tima jer smo kroz intenzivnih nekoliko tjedana rada iskusili što to znači raditi u timu na nekom projektu. Također, osjetili smo važnost dobre vremenske organiziranosti, koordiniranosti i komunikacije između članova tima. Zadovoljni smo postignutim iako vidimo prostor za poboljšanje izrađene aplikacije što je posljedica neiskustva članova tima te premalenog vremenskog okvira za izradu aplikacije.
		
		 \textit{Potrebno je točno popisati funkcionalnosti koje nisu implementirane u ostvarenoj aplikaciji.}
		
		\eject 